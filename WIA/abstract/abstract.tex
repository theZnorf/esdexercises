\documentclass[10pt,a4paper]{article}
\usepackage[latin1]{inputenc}


\begin{document}
    \title{Simulation with OMNeT++ in real time}
    \maketitle
    \section{Abstract}
    OMNeT++ embodies a framework for simulations. This includes different functionalities for communication in between modules and components written in C++.
    The normal simulation using OMNeT++ is event based and is achieved independent of the real time.
    
    OMNeT++ provides the possibility for running a simulation in real time i.e. every simulated second is processed within a real second. Such a real time simulation attempts to process the simulation with the timings and delays.
    Such a possibility to run simulated software with real timings allows the field of emulation and the connection of simulated components with real hardware or software.
    Such a real time simulation and its limits regarding possible timings depends on the used host system.
    Implementing the simulation for a given software system using OMNeT++ can be achieved with different designs and various numbers of modules and transmitted messages.
    These factors will impact the efficiency of the real time simulation and achieved timings.
    
    This paper investigates the functionality of OMNeT++ regarding simulation and real time simulation.
    
\end{document}